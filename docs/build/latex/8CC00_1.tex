%% Generated by Sphinx.
\def\sphinxdocclass{report}
\documentclass[letterpaper,10pt,english]{sphinxmanual}
\ifdefined\pdfpxdimen
   \let\sphinxpxdimen\pdfpxdimen\else\newdimen\sphinxpxdimen
\fi \sphinxpxdimen=.75bp\relax

\PassOptionsToPackage{warn}{textcomp}
\usepackage[utf8]{inputenc}
\ifdefined\DeclareUnicodeCharacter
% support both utf8 and utf8x syntaxes
  \ifdefined\DeclareUnicodeCharacterAsOptional
    \def\sphinxDUC#1{\DeclareUnicodeCharacter{"#1}}
  \else
    \let\sphinxDUC\DeclareUnicodeCharacter
  \fi
  \sphinxDUC{00A0}{\nobreakspace}
  \sphinxDUC{2500}{\sphinxunichar{2500}}
  \sphinxDUC{2502}{\sphinxunichar{2502}}
  \sphinxDUC{2514}{\sphinxunichar{2514}}
  \sphinxDUC{251C}{\sphinxunichar{251C}}
  \sphinxDUC{2572}{\textbackslash}
\fi
\usepackage{cmap}
\usepackage[T1]{fontenc}
\usepackage{amsmath,amssymb,amstext}
\usepackage{babel}



\usepackage{times}
\expandafter\ifx\csname T@LGR\endcsname\relax
\else
% LGR was declared as font encoding
  \substitutefont{LGR}{\rmdefault}{cmr}
  \substitutefont{LGR}{\sfdefault}{cmss}
  \substitutefont{LGR}{\ttdefault}{cmtt}
\fi
\expandafter\ifx\csname T@X2\endcsname\relax
  \expandafter\ifx\csname T@T2A\endcsname\relax
  \else
  % T2A was declared as font encoding
    \substitutefont{T2A}{\rmdefault}{cmr}
    \substitutefont{T2A}{\sfdefault}{cmss}
    \substitutefont{T2A}{\ttdefault}{cmtt}
  \fi
\else
% X2 was declared as font encoding
  \substitutefont{X2}{\rmdefault}{cmr}
  \substitutefont{X2}{\sfdefault}{cmss}
  \substitutefont{X2}{\ttdefault}{cmtt}
\fi


\usepackage[Bjarne]{fncychap}
\usepackage{sphinx}

\fvset{fontsize=\small}
\usepackage{geometry}


% Include hyperref last.
\usepackage{hyperref}
% Fix anchor placement for figures with captions.
\usepackage{hypcap}% it must be loaded after hyperref.
% Set up styles of URL: it should be placed after hyperref.
\urlstyle{same}

\addto\captionsenglish{\renewcommand{\contentsname}{Contents:}}

\usepackage{sphinxmessages}
\setcounter{tocdepth}{1}



\title{8CC00\_1 Documentation}
\date{Feb 27, 2021}
\release{}
\author{Rebecca Kuepper}
\newcommand{\sphinxlogo}{\vbox{}}
\renewcommand{\releasename}{}
\makeindex
\begin{document}

\pagestyle{empty}
\sphinxmaketitle
\pagestyle{plain}
\sphinxtableofcontents
\pagestyle{normal}
\phantomsection\label{\detokenize{index::doc}}



\chapter{Source Files}
\label{\detokenize{modules:source-files}}\label{\detokenize{modules::doc}}

\section{PCA module}
\label{\detokenize{AssignmentPCA:module-AssignmentPCA}}\label{\detokenize{AssignmentPCA:pca-module}}\label{\detokenize{AssignmentPCA::doc}}\index{module@\spxentry{module}!AssignmentPCA@\spxentry{AssignmentPCA}}\index{AssignmentPCA@\spxentry{AssignmentPCA}!module@\spxentry{module}}
\sphinxAtStartPar
Principal Component Analysis tools
\index{AssignmentPCA (class in AssignmentPCA)@\spxentry{AssignmentPCA}\spxextra{class in AssignmentPCA}}

\begin{fulllineitems}
\phantomsection\label{\detokenize{AssignmentPCA:AssignmentPCA.AssignmentPCA}}\pysigline{\sphinxbfcode{\sphinxupquote{class }}\sphinxcode{\sphinxupquote{AssignmentPCA.}}\sphinxbfcode{\sphinxupquote{AssignmentPCA}}}
\sphinxAtStartPar
Bases: \sphinxcode{\sphinxupquote{object}}

\sphinxAtStartPar
Class for principal component analysis of the CellLineRMAExpression data.
\index{calcLoads() (AssignmentPCA.AssignmentPCA method)@\spxentry{calcLoads()}\spxextra{AssignmentPCA.AssignmentPCA method}}

\begin{fulllineitems}
\phantomsection\label{\detokenize{AssignmentPCA:AssignmentPCA.AssignmentPCA.calcLoads}}\pysiglinewithargsret{\sphinxbfcode{\sphinxupquote{calcLoads}}}{\emph{\DUrole{n}{n}\DUrole{p}{:} \DUrole{n}{int}}, \emph{\DUrole{n}{eigpairs}\DUrole{p}{:} \DUrole{n}{list}}, \emph{\DUrole{n}{varNames}\DUrole{p}{:} \DUrole{n}{list}}}{{ $\rightarrow$ list}}
\sphinxAtStartPar
Calculate the loads of the variables on given PC.

\sphinxAtStartPar
Assumptions: 
* PC number is in range 
* eigpairs and varNames have the same length
* eigpairs and varNames are not empty
\begin{quote}\begin{description}
\item[{Parameters}] \leavevmode\begin{itemize}
\item {} 
\sphinxAtStartPar
\sphinxstyleliteralstrong{\sphinxupquote{n}} \sphinxhyphen{}\sphinxhyphen{} PC number (starting at 1).

\item {} 
\sphinxAtStartPar
\sphinxstyleliteralstrong{\sphinxupquote{eigpairs}} \sphinxhyphen{}\sphinxhyphen{} Sorted list (high\sphinxhyphen{}low) containing tuples of (eigVal, eigVec).

\item {} 
\sphinxAtStartPar
\sphinxstyleliteralstrong{\sphinxupquote{varNames}} \sphinxhyphen{}\sphinxhyphen{} List containing strings of the variable names in the same order as eigpairs.

\end{itemize}

\item[{Returns}] \leavevmode
\sphinxAtStartPar
List of (load, varName) tuples, sorted with highest load first.

\end{description}\end{quote}

\end{fulllineitems}

\index{covariance() (AssignmentPCA.AssignmentPCA method)@\spxentry{covariance()}\spxextra{AssignmentPCA.AssignmentPCA method}}

\begin{fulllineitems}
\phantomsection\label{\detokenize{AssignmentPCA:AssignmentPCA.AssignmentPCA.covariance}}\pysiglinewithargsret{\sphinxbfcode{\sphinxupquote{covariance}}}{\emph{\DUrole{n}{param1}\DUrole{p}{:} \DUrole{n}{list}}, \emph{\DUrole{n}{param2}\DUrole{p}{:} \DUrole{n}{list}}}{{ $\rightarrow$ float}}
\sphinxAtStartPar
Return the covariance of parameter lists param1 and param2.

\sphinxAtStartPar
Assumption: param1 and param2 contain numbers and are of equal length.
\begin{quote}\begin{description}
\item[{Parameters}] \leavevmode\begin{itemize}
\item {} 
\sphinxAtStartPar
\sphinxstyleliteralstrong{\sphinxupquote{param1}} \sphinxhyphen{}\sphinxhyphen{} List of parameters to be compared.

\item {} 
\sphinxAtStartPar
\sphinxstyleliteralstrong{\sphinxupquote{param2}} \sphinxhyphen{}\sphinxhyphen{} List of parameters to compare with .

\end{itemize}

\item[{Returns}] \leavevmode
\sphinxAtStartPar
covariance of param1 and param2.

\end{description}\end{quote}

\begin{sphinxVerbatim}[commandchars=\\\{\}]
\PYG{g+gp}{\PYGZgt{}\PYGZgt{}\PYGZgt{} }\PYG{n+nb+bp}{self}\PYG{o}{.}\PYG{n}{covariance}\PYG{p}{(}\PYG{p}{[}\PYG{l+m+mi}{1}\PYG{p}{,} \PYG{l+m+mi}{3}\PYG{p}{,} \PYG{l+m+mi}{5}\PYG{p}{,} \PYG{l+m+mi}{11}\PYG{p}{,} \PYG{l+m+mi}{0}\PYG{p}{,} \PYG{l+m+mi}{4}\PYG{p}{]}\PYG{p}{,} \PYG{p}{[}\PYG{l+m+mi}{2}\PYG{p}{,} \PYG{l+m+mi}{6}\PYG{p}{,} \PYG{l+m+mi}{2}\PYG{p}{,} \PYG{l+m+mi}{78}\PYG{p}{,} \PYG{l+m+mi}{1}\PYG{p}{,} \PYG{l+m+mi}{4}\PYG{p}{]}\PYG{p}{)}
\PYG{g+go}{106.4}
\PYG{g+gp}{\PYGZgt{}\PYGZgt{}\PYGZgt{} }\PYG{n+nb+bp}{self}\PYG{o}{.}\PYG{n}{covariance}\PYG{p}{(}\PYG{p}{[}\PYG{l+m+mi}{1}\PYG{p}{]}\PYG{p}{,} \PYG{p}{[}\PYG{l+m+mi}{1}\PYG{p}{,} \PYG{l+m+mi}{2}\PYG{p}{]}\PYG{p}{)}
\PYG{g+gt}{Traceback (most recent call last):}
    \PYG{o}{.}\PYG{o}{.}\PYG{o}{.}
\PYG{g+gr}{AssertionError}: \PYG{n}{Parameter lists must be of the same length.    }
\end{sphinxVerbatim}

\end{fulllineitems}

\index{cumulativeMovingAverage() (AssignmentPCA.AssignmentPCA method)@\spxentry{cumulativeMovingAverage()}\spxextra{AssignmentPCA.AssignmentPCA method}}

\begin{fulllineitems}
\phantomsection\label{\detokenize{AssignmentPCA:AssignmentPCA.AssignmentPCA.cumulativeMovingAverage}}\pysiglinewithargsret{\sphinxbfcode{\sphinxupquote{cumulativeMovingAverage}}}{\emph{\DUrole{n}{x}\DUrole{p}{:} \DUrole{n}{list}}}{{ $\rightarrow$ list}}
\sphinxAtStartPar
Return a list of the cumulative moving average of input parameterlist x.

\sphinxAtStartPar
Assumption: x contains numbers or is empty.
\begin{quote}\begin{description}
\item[{Parameters}] \leavevmode
\sphinxAtStartPar
\sphinxstyleliteralstrong{\sphinxupquote{x}} \sphinxhyphen{}\sphinxhyphen{} list of numbers.

\item[{Returns}] \leavevmode
\sphinxAtStartPar
list of float cumulative moving averages.

\end{description}\end{quote}

\begin{sphinxVerbatim}[commandchars=\\\{\}]
\PYG{g+gp}{\PYGZgt{}\PYGZgt{}\PYGZgt{} }\PYG{n+nb+bp}{self}\PYG{o}{.}\PYG{n}{cumulativeMovingAverage}\PYG{p}{(}\PYG{p}{[}\PYG{l+m+mi}{1}\PYG{p}{,} \PYG{l+m+mi}{3}\PYG{p}{,} \PYG{l+m+mi}{5}\PYG{p}{,} \PYG{l+m+mi}{11}\PYG{p}{,} \PYG{l+m+mi}{0}\PYG{p}{,} \PYG{l+m+mi}{4}\PYG{p}{]}\PYG{p}{)}
\PYG{g+go}{[1.0, 2.0, 3.0, 5.0, 4.0, 4.0]}
\PYG{g+gp}{\PYGZgt{}\PYGZgt{}\PYGZgt{} }\PYG{n+nb+bp}{self}\PYG{o}{.}\PYG{n}{cumulativeMovingAverage}\PYG{p}{(}\PYG{p}{[}\PYG{p}{]}\PYG{p}{)}
\PYG{g+go}{[]}
\end{sphinxVerbatim}

\end{fulllineitems}

\index{listOfCellLineNumbers (AssignmentPCA.AssignmentPCA attribute)@\spxentry{listOfCellLineNumbers}\spxextra{AssignmentPCA.AssignmentPCA attribute}}

\begin{fulllineitems}
\phantomsection\label{\detokenize{AssignmentPCA:AssignmentPCA.AssignmentPCA.listOfCellLineNumbers}}\pysigline{\sphinxbfcode{\sphinxupquote{listOfCellLineNumbers}}\sphinxbfcode{\sphinxupquote{ = {[}1, 2, 3, 4, 5, 6, 7, 8, 9, 10, 11, 12, 13, 14, 15, 16, 17, 18, 19, 20, 21, 22, 23, 24, 25, 26, 27, 28, 30, 31, 32, 33, 34, 35, 36, 37, 38, 39, 40, 42, 43, 44, 45, 46, 47, 48, 49, 50, 51, 52, 53, 54, 55, 56, 57, 58, 59, 60, 61, 62, 63, 64, 65, 66, 67, 68, 69, 70, 71, 72, 73, 74, 75, 76, 77, 78, 79, 80, 81, 82, 83, 84, 85, 86, 87, 88, 89, 90, 91, 92, 93, 94, 95, 96, 97, 98, 99, 100, 101, 102, 103, 104, 105, 106, 107, 108, 109, 110, 111, 112, 113, 114, 115, 116, 117, 118, 119, 120, 121, 122, 123, 124, 125, 126, 127, 128, 129, 130, 131, 132, 133, 134, 135, 136, 137, 138, 139, 140, 141, 142, 143, 144, 145, 146, 147{]}}}}
\end{fulllineitems}

\index{plotCumulativeMovingAverage() (AssignmentPCA.AssignmentPCA method)@\spxentry{plotCumulativeMovingAverage()}\spxextra{AssignmentPCA.AssignmentPCA method}}

\begin{fulllineitems}
\phantomsection\label{\detokenize{AssignmentPCA:AssignmentPCA.AssignmentPCA.plotCumulativeMovingAverage}}\pysiglinewithargsret{\sphinxbfcode{\sphinxupquote{plotCumulativeMovingAverage}}}{\emph{\DUrole{n}{x}\DUrole{p}{:} \DUrole{n}{list}}, \emph{\DUrole{n}{title}\DUrole{p}{:} \DUrole{n}{str} \DUrole{o}{=} \DUrole{default_value}{\textquotesingle{}Cumulative moving average\textquotesingle{}}}}{{ $\rightarrow$ None}}
\sphinxAtStartPar
Plot the cumulative moving average of a list x

\sphinxAtStartPar
Assumption: x contains numbers or is empty.
\begin{quote}\begin{description}
\item[{Parameters}] \leavevmode\begin{itemize}
\item {} 
\sphinxAtStartPar
\sphinxstyleliteralstrong{\sphinxupquote{x}} \sphinxhyphen{}\sphinxhyphen{} List of parameters to be calculated moving average of and plotted.

\item {} 
\sphinxAtStartPar
\sphinxstyleliteralstrong{\sphinxupquote{title}} \sphinxhyphen{}\sphinxhyphen{} string containing a title for the graph.

\end{itemize}

\end{description}\end{quote}

\end{fulllineitems}

\index{readRMAExpressionAssigned() (AssignmentPCA.AssignmentPCA method)@\spxentry{readRMAExpressionAssigned()}\spxextra{AssignmentPCA.AssignmentPCA method}}

\begin{fulllineitems}
\phantomsection\label{\detokenize{AssignmentPCA:AssignmentPCA.AssignmentPCA.readRMAExpressionAssigned}}\pysiglinewithargsret{\sphinxbfcode{\sphinxupquote{readRMAExpressionAssigned}}}{}{{ $\rightarrow$ list}}
\sphinxAtStartPar
Return a list of RMA expression values of assigned cell lines stored 
in self.listOfCellLineNumbers.

\end{fulllineitems}


\end{fulllineitems}



\section{cell line module}
\label{\detokenize{CellLineRMAExpression:module-CellLineRMAExpression}}\label{\detokenize{CellLineRMAExpression:cell-line-module}}\label{\detokenize{CellLineRMAExpression::doc}}\index{module@\spxentry{module}!CellLineRMAExpression@\spxentry{CellLineRMAExpression}}\index{CellLineRMAExpression@\spxentry{CellLineRMAExpression}!module@\spxentry{module}}
\sphinxAtStartPar
Analysis of dataset. (RMA = Rombust Multi\sphinxhyphen{}array Averages)
\index{CellLineRMAExpression (class in CellLineRMAExpression)@\spxentry{CellLineRMAExpression}\spxextra{class in CellLineRMAExpression}}

\begin{fulllineitems}
\phantomsection\label{\detokenize{CellLineRMAExpression:CellLineRMAExpression.CellLineRMAExpression}}\pysiglinewithargsret{\sphinxbfcode{\sphinxupquote{class }}\sphinxcode{\sphinxupquote{CellLineRMAExpression.}}\sphinxbfcode{\sphinxupquote{CellLineRMAExpression}}}{\emph{\DUrole{n}{type}\DUrole{p}{:} \DUrole{n}{str}}}{}
\sphinxAtStartPar
Bases: \sphinxcode{\sphinxupquote{object}}

\sphinxAtStartPar
Class for analysis of cancer cell data.
\index{cancerType() (CellLineRMAExpression.CellLineRMAExpression method)@\spxentry{cancerType()}\spxextra{CellLineRMAExpression.CellLineRMAExpression method}}

\begin{fulllineitems}
\phantomsection\label{\detokenize{CellLineRMAExpression:CellLineRMAExpression.CellLineRMAExpression.cancerType}}\pysiglinewithargsret{\sphinxbfcode{\sphinxupquote{cancerType}}}{\emph{\DUrole{n}{cellLine}\DUrole{p}{:} \DUrole{n}{str}}}{{ $\rightarrow$ str}}
\sphinxAtStartPar
Return the name of the cancer type for a given cell line.

\sphinxAtStartPar
Assumption: cellLine exists.
\begin{quote}\begin{description}
\item[{Parameters}] \leavevmode
\sphinxAtStartPar
\sphinxstyleliteralstrong{\sphinxupquote{cellLine}} \textendash{} String containing the name of a cell line.

\item[{Returns}] \leavevmode
\sphinxAtStartPar
String containing the name of the type of cancer with which the cell line is associated, or None if cellLine doesn’t exist.

\end{description}\end{quote}

\begin{sphinxVerbatim}[commandchars=\\\{\}]
\PYG{g+gp}{\PYGZgt{}\PYGZgt{}\PYGZgt{} }\PYG{n+nb+bp}{self}\PYG{o}{.}\PYG{n}{cancerType}\PYG{p}{(}\PYG{l+s+s1}{\PYGZsq{}}\PYG{l+s+s1}{AU565}\PYG{l+s+s1}{\PYGZsq{}}\PYG{p}{)}
\PYG{g+go}{BRCA}
\PYG{g+gp}{\PYGZgt{}\PYGZgt{}\PYGZgt{} }\PYG{n+nb+bp}{self}\PYG{o}{.}\PYG{n}{cancerType}\PYG{p}{(}\PYG{l+s+s1}{\PYGZsq{}}\PYG{l+s+s1}{\PYGZsq{}}\PYG{p}{)}
\PYG{g+go}{None}
\end{sphinxVerbatim}

\end{fulllineitems}

\index{readRMAExpression() (CellLineRMAExpression.CellLineRMAExpression method)@\spxentry{readRMAExpression()}\spxextra{CellLineRMAExpression.CellLineRMAExpression method}}

\begin{fulllineitems}
\phantomsection\label{\detokenize{CellLineRMAExpression:CellLineRMAExpression.CellLineRMAExpression.readRMAExpression}}\pysiglinewithargsret{\sphinxbfcode{\sphinxupquote{readRMAExpression}}}{\emph{\DUrole{n}{cellLine}\DUrole{p}{:} \DUrole{n}{str}}}{{ $\rightarrow$ list}}
\sphinxAtStartPar
Read the RMA expression of a single cell line.

\sphinxAtStartPar
Assumption: cellLine in dataset.
\begin{quote}\begin{description}
\item[{Parameters}] \leavevmode
\sphinxAtStartPar
\sphinxstyleliteralstrong{\sphinxupquote{cellLine}} \textendash{} String, cell line from which the RMA expression is to be read.

\item[{Returns}] \leavevmode
\sphinxAtStartPar
list of RMA expressions of all 244 genes of cellLine, or None if cellLine does not exist.

\end{description}\end{quote}

\begin{sphinxVerbatim}[commandchars=\\\{\}]
\PYG{g+gp}{\PYGZgt{}\PYGZgt{}\PYGZgt{} }\PYG{n+nb+bp}{self}\PYG{o}{.}\PYG{n}{readRMAExpression}\PYG{p}{(}\PYG{l+s+s1}{\PYGZsq{}}\PYG{l+s+s1}{\PYGZsq{}}\PYG{p}{)}
\PYG{g+go}{None}
\PYG{g+gp}{\PYGZgt{}\PYGZgt{}\PYGZgt{} }\PYG{n+nb}{len}\PYG{p}{(}\PYG{n+nb+bp}{self}\PYG{o}{.}\PYG{n}{readRMAExpression}\PYG{p}{(}\PYG{l+s+s1}{\PYGZsq{}}\PYG{l+s+s1}{AU565}\PYG{l+s+s1}{\PYGZsq{}}\PYG{p}{)}\PYG{p}{)}
\PYG{g+go}{244}
\end{sphinxVerbatim}

\end{fulllineitems}


\end{fulllineitems}



\section{main module}
\label{\detokenize{main:module-main}}\label{\detokenize{main:main-module}}\label{\detokenize{main::doc}}\index{module@\spxentry{module}!main@\spxentry{main}}\index{main@\spxentry{main}!module@\spxentry{module}}
\sphinxAtStartPar
Main file for PCA Analysis of cancer cell RNA expression of different genes.

\sphinxAtStartPar
In this modules, the following steps were taken to perform PCA:
\begin{enumerate}
\sphinxsetlistlabels{\arabic}{enumi}{enumii}{}{.}%
\item {} 
\sphinxAtStartPar
The data was read and converted to a np array.

\item {} 
\sphinxAtStartPar
The covariance of the genes was calculated using the \sphinxcode{\sphinxupquote{covariance()}} function from the {\hyperref[\detokenize{AssignmentPCA:module-AssignmentPCA}]{\sphinxcrossref{\sphinxcode{\sphinxupquote{AssignmentPCA}}}}} module.

\item {} 
\sphinxAtStartPar
From this matrix, the eigenvalues and eigenvectors were calculated and sorted by relevance.

\item {} 
\sphinxAtStartPar
The cumulative overall variance was then calculated.

\item {} 
\sphinxAtStartPar
A bar graph of principal component contribution and a scree plot were created.

\item {} 
\sphinxAtStartPar
It was decided to reduce to 3 dimensions.

\item {} 
\sphinxAtStartPar
A new subspace with only 3 PC’s was created.

\item {} 
\sphinxAtStartPar
A 2D and 3D principal component plot were generated.

\item {} 
\sphinxAtStartPar
Loads for the three major PC’s were calculated using \sphinxcode{\sphinxupquote{calcLoads()}} from {\hyperref[\detokenize{AssignmentPCA:module-AssignmentPCA}]{\sphinxcrossref{\sphinxcode{\sphinxupquote{AssignmentPCA}}}}}.

\item {} 
\sphinxAtStartPar
For each of the three major PC’s, the 50 highest loads of the genes has been shown in a bar graph.

\end{enumerate}


\chapter{Indices and tables}
\label{\detokenize{index:indices-and-tables}}\begin{itemize}
\item {} 
\sphinxAtStartPar
\DUrole{xref,std,std-ref}{genindex}

\item {} 
\sphinxAtStartPar
\DUrole{xref,std,std-ref}{modindex}

\item {} 
\sphinxAtStartPar
\DUrole{xref,std,std-ref}{search}

\end{itemize}


\renewcommand{\indexname}{Python Module Index}
\begin{sphinxtheindex}
\let\bigletter\sphinxstyleindexlettergroup
\bigletter{a}
\item\relax\sphinxstyleindexentry{AssignmentPCA}\sphinxstyleindexpageref{AssignmentPCA:\detokenize{module-AssignmentPCA}}
\indexspace
\bigletter{c}
\item\relax\sphinxstyleindexentry{CellLineRMAExpression}\sphinxstyleindexpageref{CellLineRMAExpression:\detokenize{module-CellLineRMAExpression}}
\indexspace
\bigletter{m}
\item\relax\sphinxstyleindexentry{main}\sphinxstyleindexpageref{main:\detokenize{module-main}}
\end{sphinxtheindex}

\renewcommand{\indexname}{Index}
\printindex
\end{document}